\section{Kylo and DNA}
	\subsection{Statement}
		Given a string $S$ over the alphabet
		$\{\,\ttt{A}, \ttt{G}, \ttt{C}, \ttt{T}\,\}$ and another string
		$T$ over $\{\,\ttt{A}, \ttt{G}, \ttt{C}, \ttt{T}, \ttt{.}\,\}$
		find all the indices of $S$ such that $T$ matches the substring
		of $S$ starting at that index.

		The character \ttt{.} matches every character in the alphabet of $S$.

	\subsection{A simpler version}
		Let's remove the \ttt{.} from the alphabet of $T$.
		Then we have the classic string matching problem.

		Of course, there are some very well known algorithms for this (easy), but it
		is not easy to extend them to the original problem, so we solve this variant
		using a technique that is harder, but generalises easily to the problem we
		wish to solve.

		For convenience, we assume that both $S$ and $T$ are \ttt{0-indexed}, and define
		$A = \{\,\ttt{A}, \ttt{G}, \ttt{C}, \ttt{T}\,\}$.

		Define $\ttt{match}(c, i)$ as $\card{\{\, j\mid T_j = S_{i + j}, T_j = c\,\}}$.

		\begin{claim}
		$T$ matches $S_{i\dots i + \card{T} - 1} \iff \sum_{c \in A} \ttt{match}(c, i) = \card{T}$.
		\end{claim}

		\begin{proof}
		Left as an exercise $\ddot\smallsmile$.
		\end{proof}

		Now, of course, we need to find $\ttt{match}(c, i) \forall c\in A,
		\forall i, 0 \leq i < N$.

		Let's concentrate on a single $c \in A$.

		Denote by $f(S, c)$ the polynomial
		$\sum_{i = 0}^{\card{S} - 1} \left[S_i = c\right] x^i$.
		(the coefficient of $x^i$ is $1 \iff S_i = c$).

		What happens if we take the product $f(S, c) \times f(T, c)$?
		We get the polynomial
		\begin{align*}
			f(S, c) \times f(T, c) &=
			\sum_{i = 0}^{\card{S} + \card{T} - 2} \left(\sum_{j = 0}^{\min(i, \card{T} - 1)}
			\left[S_{i - j} = c\right]\left[T_{j} = c\right] x^i\right)\\
		\end{align*}

		Now, we seem to be matching the reverse of $T$ with $S$ in some fashion, so
		let's try reversing one (I choose $S$, let's call the reverse $S'$), and then
		computing the product $f(S', c) \times f(T, c)$.
		We get the polynomial

		\begin{align*}
			f(S', c) \times f(T, c) &= 
			\sum_{i = 0}^{\card{S'} + \card{T} - 2} \left(\sum_{j = 0}^{\min(i, \card{T} - 1)}
			\left[{S'}_{i - j} = c\right]\left[T_{j} = c\right] x^i\right)\\
			&= \sum_{i = 0}^{\card{S'} + \card{T} - 2} \left(\sum_{j = 0}^{\min(i, \card{T} - 1)}
			\left[{S}_{\card{S} - 1 - i + j} = c\right]\left[T_{j} = c\right] x^i \right)\\
		\end{align*}

		\begin{claim}
			$\ttt{match}(c, i) = \left[x^{\card{S} - i - 1}\right] f(S', c)\times f(T, c)$
			(that is, the coefficient of $x^{\card{S} - i - 1}$ in $f(S', c) \times f(T, c)$.
		\end{claim}

		\begin{proof}
			Left as an exercise.
		\end{proof}

		We can compute the polynomials we want in linear time, and the product in
		$\bigO{(\card{S} + \card{T}) \log (\card{S} + \card{T})}$ time.
		This is also multiplied by a factor of $\card{A}$, because we do this once
		for each letter of the alphabet.

	\subsection{Extending solution to original problem}
		The only change we make is to define $f(S, c) = 
		\sum_{i = 0}^{\card{S} - 1} \left[S_i = \ttt{.} \vee S_i = c\right] x^i$.
		(the coefficient of $x_i$ is $1 \iff S_i = \ttt{.} \vee S_i = c$).

		You should try to work out the details and see that solution still works.

	\subsection{A faster, more general solution}
		Are you annoyed by that pesky $\card{A}$ factor in the solution?
		Then this is the subsection for you!

		Replace each letter of your alphabet $A$ by the numbers $1, \dots , \card{A}$,

		\begin{claim}
			$S_{i\dots i + \card{T} - 1}$ and $T$ match $\iff
			\sum_{j = 0}^{\card{T} - 1} \left(S_{i + j} - T_{j}\right) ^ 2 = 0$.
		\end{claim}

		\begin{proof}
			Trivial.
		\end{proof}

		Now, we open up the summand to get
		\begin{align*}
			S_{i\dots i + \card{T} - 1} \text{ and } T \ttt{ match }
			&\iff
			\sum_{j = 0}^{\card{T} - 1} \left(S_{i + j} - T_{j}\right) ^ 2 = 0 \\
			&\iff
			\sum_{j = 0}^{\card{T} - 1} S_{i + j}^2 + T_{j}^2 - 2S_{i + j}T_j = 0\\
			&\iff
			\sum_{j = 0}^{\card{T} - 1} S_{i +j}^2 + 
			\sum_{j = 0}^{\card{T} - 1} T_j^2 -
			2\sum_{j = 0}^{\card{T} - 1} S_{i + j}T_j = 0\\
		\end{align*}

		Now, the first summation is just a range-sum query over the array
		$X$, where $X_i = \left(S_i\right)^2$ (you can do in O(1) using prefix sums),
		and the second summation is a prefix of a similar array for $T$, so also O(1).

		The last summation, is actually a form exactly similar to $\ttt{match}(c, i)$,
		and can, in fact, be represented as a convolution of $S'$ and $T$.

		How do we extend this to where $T$ can have \ttt{.}s?
		Simple, we give \ttt{.} the number $0$, and everything else same as before,
		we now check whether 
		$\sum_{j = 0}^{\card{T} - 1} \left(S_{i + j} - T_{j}\right)^2T_j = 0$.

		The effect of multiplying with $T_j$ is that the term gets eliminated straight
		away if $T_j = 0$, and otherwise stays non-zero / zero depending on the term
		inside the square, which is as desired.

		This approach is more general, because what if $S$ had dots too?
		Simple, we check whether
		$\sum_{j = 0}^{\card{T} - 1} \left(S_{i + j} - T_{j}\right)^2T_jS_{i + j} = 0$.

		Of course, after opening the square in either case above, you will get something you
		can convert to a convolution, but I leave that as exercise.

		(I learned this trick from a lecture by
		\href{http://codeforces.com/profile/kevinsogo}{Kevin Charles Atienza})

	\subsection{A note on implementation}
		During the contest, I saw that a participant submitted a recursive FFT implementation.
		It did not pass the 10 second TL, while my testing solution (which used an iterative FFT)
		ran in 3 seconds.

		The recursive solution (which was otherwise correct) actually took 17 seconds to pass the
		largest tests. While I will keep this in mind for any subsequent FFT problems I make
		(and test with recursive solution myself), you will often find that some contest does not
		allow recursive FFT to pass.

		I highly recommend that you go through the section on iterative FFT in CLRS, and then implement
		your own (\href{https://github.com/parthmittal/codebook/blob/master/fastfft.cpp}{here} is mine
		for reference).

		Also \href{https://pastebin.com/Jf7H8Prp}{here} is a link to my full solution; as an optimisation challenge,
		can you reduce its runtime by a factor of $4$ (max-test is
		\href{https://drive.google.com/file/d/0B-W-TWxgtybGQzhvR0hGZUM2MU0/view?usp=sharing}{here})?
		(technically you should be able to reduce it by a factor of around 8x, by using the previous section's observations too)

		Some ideas: replace \verb|std::complex| by a hand-rolled complex numbers class, and note that the transform
		steps are expensive, so maybe you want to minimise the number of those?
